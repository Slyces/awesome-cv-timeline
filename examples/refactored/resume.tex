% TEX program = xelatex
% ------------------------------------------------------------------------------
% Important note: this template must be compiled with XeLaTeX; the below lines
% will ensure this
%!TEX TS-program = xelatex
%!TEX encoding = UTF-8 Unicode
% ------------------------------------------------------------------------------
\documentclass[11pt, a4paper]{../../resume}

% Enable to see border on tables and other layout indicators
\setbool{debugmode}{false}

% Configure page margins with geometry
\geometry{left=1.5cm, top=1.5cm, right=1.5cm, bottom=2cm, footskip=.5cm}

% Color for highlights
\colorlet{awesome}{awesome-darkblue}

% Add multiple .bib files
\addbibresource{bibliographies/papers-chapters.bib}
\addbibresource{bibliographies/patents.bib}

% Set lenghts for the whole document
\renewcommand{\headerafterskip}{1mm}
\renewcommand{\summaryafterskip}{1mm}
\renewcommand{\sectioninterskip}{2mm}

% Header & Contacts Information
% --------------------------------------
\name{Jack}{Sparrow}
\position{Captain{\enskip\cdotp\enskip}Pirate{\enskip\cdotp\enskip}Coach}
\photo[circle,edge,left]{images/jack.jpg}
\setlength{\headerphotowidth}{3.8cm}

\address{Port Tortuga, Caribbean Sea}
\mobile{+1-800-SEABOUND}
\email{pirateking@blackpearl.com}
\github{jspgit}
\linkedin{jacksparrow}
\orcid{1701-0420-1720-0800}

% Perform some setup for debugging displays
\setupdebug

% ------------------------------------------------------------------------------

\begin{document}

% The header is entirely defined by the definitions above
\makecvheader[C]

% Specify the letter footer with 3 arguments: (<left>, <center>, <right>), leave
% any of these blank if they are not needed
\makecvfooter{\thepage}{Jack Sparrow~~~·~~~Resume}{\today}

% Short summary about yourself
\begin{cvsummary}
    To continue a life of adventure, sailing the high seas in search of
    treasure, rum, and the occasional encounter with supernatural forces, while
    maintaining the reputation as the most cunning and unpredictable pirate in
    the Caribbean.
\end{cvsummary}

% ------------------------------------------------------------------------------
% Page 1 - Dual column layout

% Set lenghts for the first page (double column layout)
\setlength{\leftcolumnwidth}{5cm}
\setlength{\leftcolumnrightmargin}{0.5cm}
\setlength{\timelinemargin}{0.4cm}

% Draw the timeline for experience & education (4th & 5th sections)
\timeline{D,E}

% Textblock for the left column
\begin{textblock*}{\leftcolumnwidth}(0pt, 0pt)
    \begin{debugframe}
        %----------------------------------------------------------------------------------------
%	CONTACT
%----------------------------------------------------------------------------------------

\cvsection{Contact}

\begin{tabular*}{\leftcolumnwidth - 2\sidemargin + 2\iconswidth}{c@{}C{0.45\leftcolumnwidth - 0.5\sidemargin + 0.5\iconswidth}@{}c@{}R{0.55\leftcolumnwidth - 0.50\sidemargin + \iconswidth}}
    \caddressicon & \multicolumn{3}{r}{\caddress}\\
    \cmobileicon & \multicolumn{3}{r}{\cmobile}\\
    \cemailicon & \multicolumn{3}{r}{\cemail}\\
    \cgiticon & & &\cgituser \\
    \clinkedinicon & & & \clinkedinuser\\
    \corcidicon & \multicolumn{3}{r}{\corciduser}\\
    % \faHome & & & Marital status: married
\end{tabular*}

        %----------------------------------------------------------------------------------------
%	SECTION TITLE
%----------------------------------------------------------------------------------------

\cvsection{Skills}

\begin{cvskills}

% ------------------------------------------------
\skillcategory{Programming Languages}

\begin{skillstable}
    \skill{Rust}\skill{C++}\skill{Java}
\end{skillstable}

% ------------------------------------------------
\skillcategory{Java Stack}

\begin{skillstable}
    \skill{Framework}\skill{Library}\skill{Framework}
\end{skillstable}

% ------------------------------------------------
\skillcategory{Databases / Services}

\begin{skillstable}
    \skill{PostgreSQL}\skill{MongoDB}\skill{Redis}\\[3pt]
    \skill{Kafka}
\end{skillstable}

% ------------------------------------------------
\skillcategory{Tools}

\begin{skillstable}
    \skill{Git}\skill{Bash}\skill{Docker}
\end{skillstable}

% ------------------------------------------------
\skillcategory{Software}

\begin{skillstable}
    \skill{Word}\skill{Latex}\skill{Powerpoint}
\end{skillstable}

% ------------------------------------------------
\skillcategory{Languages}

\vspace{-1mm}
\begin{skillstable}
    \languagestyle{Esperanto \textit{(native)}, English \textit{(fluent)}}
\end{skillstable}

\end{cvskills}

\vspace{2mm}

        %----------------------------------------------------------------------------------------
%	HOBBIES
%----------------------------------------------------------------------------------------

\cvsection{Hobbies}

\begin{tabular*}{\leftcolumnwidth - \sidemargin}{cR{\leftcolumnwidth - \sidemargin + \iconswidth}@{}}
    \hobbyiconstyle{\icon{󱗖}} & \hobbiesstyle{Ducimus qui blanditiis} \\
    \hobbyiconstyle{\icon{}} & \hobbiesstyle{Quidem rerum facilis est et} \\
    \hobbyiconstyle{\icon{}} & \hobbiesstyle{Minus id quod maxime placeat} \\
    \hobbyiconstyle{\icon{󰇦}} & \hobbiesstyle{Temporibus autem quibusdam} \\
\end{tabular*}

    \end{debugframe}
\end{textblock*}

% Textblock for the main body (right column)
\begin{textblock*}
    {\textwidth - \leftcolumnwidth - \leftcolumnrightmargin - 2\timelinemargin}
    (\leftcolumnwidth + \leftcolumnrightmargin + 2\timelinemargin, 0pt)
    \begin{debugframe}
        %----------------------------------------------------------------------------------------
%	SECTION TITLE
%----------------------------------------------------------------------------------------

\cvsection{Experience}

%----------------------------------------------------------------------------------------
%	SECTION CONTENT
%----------------------------------------------------------------------------------------

\begin{cventries}

%------------------------------------------------

\cventry
{Great Company Inc.} % Job title
{Software Engineer} % Organization
{London, UK} % Location
{Dec. 2017 - PRESENT\duration{5.5 years}} % Date(s)
{
\begin{cvitems}
\setlength{\itemsep}{1ex} %
%
% ---
\item {Sed ut perspiciatis unde omnis iste natus error sit voluptatem accusantium}
    \begin{cvitemsnested}
    \item {Doloremque laudantium, totam rem aperiam, eaque ipsa}
    \item {Ab illo inventore veritatis et quasi architecto beatae vitae dicta sunt explicabo}
    \end{cvitemsnested}
%
% ---
\item {Enim ipsam voluptatem quia voluptas sit aspernatur}
    \begin{cvitemsnested}
    \item {Sed quia consequuntur magni dolores eos qui ratione}
    \item {Voluptatem sequi nesciunt. Neque porro quisquam}
    \end{cvitemsnested}
%
% ---
\item {Responsible of the main backend component handling the ML data pipeline}
    \begin{cvitemsnested}
    \item {REST API using \keyword{FastAPI} \& \keyword{PostgresSQL}}
    \item {Stores ML pipeline data (feature engineering, model training)}
    \item {Serves explainability graphs and other derivative data to frontend}
    \end{cvitemsnested}
%
% ---
\item {Qui dolorem ipsum quia dolor sit amet}
    \begin{cvitemsnested}
    \item {Consectetur, adipisci velit, sed quia non numquam eius modi tempora}
    \item {Ut enim ad minima veniam, quis nostrum ullam}
    \end{cvitemsnested}
%
\end{cvitems}
}

%------------------------------------------------

\cventry
{Smaller Company} % Job title
{Intern - Computer Stuff} % Organization
{Miami, Germany} % Location
{Jan. 2017 - Aug. 2017\duration{7 months}} % Date(s)
{ % Description(s) of tasks/responsibilities
\begin{cvitems}
\item {Exercitationem ullam corporis suscipit laboriosam, nisi}
\item {Quis autem vel eum iure reprehenderit qui in ea voluptate velit}
\item {esse quam nihil molestiae consequatur, vel illum qui dolorem eum fugiat quo}
\end{cvitems}
}

%------------------------------------------------

\cventry
{Some University} % Organization
{Laboratory Intern} % Job title
{Bordeaux, France} % Location
{Jun. 2016 - Jul. 2016\duration{2 months}} % Date(s)
{ % Description(s) of tasks/responsibilities
\begin{cvitems}
\item {Et iusto odio dignissimos ducimus qui blanditiis praesentium}
\item {Et harum quidem rerum facilis est et expedita distinctio. Nam libero}
\item {Minus id quod maxime placeat facere possimus}
\end{cvitems}
}

%------------------------------------------------

\end{cventries}

        %----------------------------------------------------------------------------------------
%	SECTION TITLE
%----------------------------------------------------------------------------------------

\cvsection{Education}

%----------------------------------------------------------------------------------------
%	SECTION CONTENT
%----------------------------------------------------------------------------------------

\begin{cventries}

%------------------------------------------------

%\newcommand*\honourstyle[1]{\fontsize{10pt}{1em}\bodyfontlight\itshape\color{text} #1}

\cventry
{Master's Degree. in Computer Science (Year 2)} % Degree
{Great University} % Institution
{Paris VII, France} % Location
{Sept. 2016 - Sept. 2017} % Date(s)
{ % Description(s) bullet points
    Autem quibusdam et aut officiis debitis aut rerum necessitatibus
}

%------------------------------------------------

\end{cventries}

\begin{cventries}

%------------------------------------------------

\cventry
{Master's Degree. in Computer Science (Year 1)} % Degree
{Awesome University} % Institution
{Frankfurt, Germany} % Location
{Sept. 2015 - June. 2016} % Date(s)
{ % Description(s) bullet points
    Saepe eveniet ut et voluptates
}

%------------------------------------------------

\end{cventries}

\begin{cventries}

%------------------------------------------------

\cventry
{Bachelor's Degree. in Mathematics, Computer Science} % Degree
{Awesome University} % Institution
{Frankfurt, Germany} % Location
{Sept. 2012 - June. 2015} % Date(s)
{ % Description(s) bullet points
    Saepe eveniet ut et voluptates
}

%------------------------------------------------

\end{cventries}

    \end{debugframe}
\end{textblock*}

% New page
\newpage

% Set the left column lengths to draw the timeline on the left
\setlength{\leftcolumnwidth}{0cm}
\setlength{\leftcolumnrightmargin}{0cm}

% Draw the timeline for projects & selected publications (6th & 7th sections)
\timeline{F,G}

% Shift the content to the right to have space for the timeline
\begin{textblock*}{\textwidth - 2\timelinemargin}(2\timelinemargin, 0pt)
    \begin{debugframe}
        %----------------------------------------------------------------------------------------
%	SECTION TITLE
%----------------------------------------------------------------------------------------

\cvsection{Projects}

%----------------------------------------------------------------------------------------
%	SECTION CONTENT
%----------------------------------------------------------------------------------------

\begin{cventries}
%------------------------------------------------

\cvproject
{Multi Agents Attraction/Repulsion} % Project title
{3 months - 2016} % Duration
{ % Description(s) of tasks/responsibilities
    Multi Agents, Article Implementation
}

%------------------------------------------------

%\cvproject
%{Procedural Planet Generation} % Project title
%{5 months - 2017} % Duration
%{ % Description(s) of tasks/responsibilities
    %Voronoï Diagrams, Graphs, 3D render
%}

%%------------------------------------------------

\cvproject
{Artificial Chess Players} % Project title
{3 months - 2016} % Duration
{ % Description(s) of tasks/responsibilities
    MinMax, NegaMax, Q-Learning
}

%------------------------------------------------

\cvproject
{Prediction of Market Fluctuations} % Project title
{4 months - 2016} % Duration
{ % Description(s) of tasks/responsibilities
    Supervised Task, Deep Learning
}

%------------------------------------------------

\end{cventries}

        %----------------------------------------------------------------------------------------
%	SECTION TITLE
%----------------------------------------------------------------------------------------

\cvsection{Selected publications}

%----------------------------------------------------------------------------------------
%	SECTION CONTENT
%----------------------------------------------------------------------------------------
\begin{cventries}
\begin{refsection}[bibliographies/papers-chapters.bib]
\setlength{\bibhang}{0pt}
    \descriptionstyle{%
    \nocite{Sparrow1715,Sparrow1720,Sparrow1718,Sparrow1722}
    \printbibliography[heading=none]
    }
\end{refsection}
\end{cventries}


    \end{debugframe}
\end{textblock*}

\newpage

% No timeline for full publications
\begin{debugframe}
%----------------------------------------------------------------------------------------
%	SECTION TITLE
%----------------------------------------------------------------------------------------

\cvsection{Publications}

%----------------------------------------------------------------------------------------
%	SECTION CONTENT
%----------------------------------------------------------------------------------------
\entrytitlestyle{Peer-reviewed publications}
\begin{refsection}[bibliographies/papers-chapters.bib]
    \descriptionstyle{
    \nocite{*}
    \printbibliography[heading=none]
    }
\end{refsection}

%----------------------------------------------------------------------------------------
%	SECTION CONTENT
%----------------------------------------------------------------------------------------
\entrytitlestyle{Patents}
\begin{refsection}[bibliographies/patents.bib]
    \descriptionstyle{
    \nocite{*}
    \printbibliography[heading=none]
    }
\end{refsection}

\end{debugframe}


\end{document}
